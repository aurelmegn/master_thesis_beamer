%définir ce que c'est qu'un rôle
%faire le schéma de rôles
%
%définir ce que c'est qu'un agent
%montrer le schéma de rôles
%
%revoir les définitions de rôle, agents
%
%mettre l'image de l'architecture sur le côté


\section{Introduction}\label{sec:section-1}
\begin{frame}
    \frametitle{Introduction}
    \framesubtitle{Contexte}
    Selon le rapport~\autocite{somik2017opening}, la population des villes africaines est en forte croissance.

    \begin{itemize}
        \item La forte occupation des villes augmente la demande de déplacement et influe sur leur développement.
        \item Un autre aspect influant le développement d'une ville est la qualité de la capacité de déplacement des personnes et des biens marchands.
        \item La fluidité des déplacements est impactée par les règles de régulation du traffic routier et par les changements au niveau des infrastructures routières.
    \end{itemize}

\end{frame}

\begin{frame}
    \frametitle{Introduction}
    \framesubtitle{Problématique}

    \begin{itemize}
        \item Les infrastructures routières et les outils de régulation du traffic doivent évoluer pour supporter la demande croissante de mobilité.
        \item Pour prendre des décisions sur l'évolution de ces infrastructures, il faut pouvoir évaluer l'impact sur la qualité de la mobilité :
        \begin{itemize}
            \item de l'évolution des infrastructures routières
            \item des changements des règles de régulation routières
        \end{itemize}
    \end{itemize}

    \begin{alertblock}{Problème}

        \begin{itemize}
            \item C'est un travail fastidieux qui peut rapidement se complexifier
            \item Ce n'est pas évident, voir impossible de le faire manuellement
        \end{itemize}
    \end{alertblock}
\end{frame}

\begin{frame}
    \frametitle{Introduction}
    \framesubtitle{Objectifs}
    Mettre en place un simulateur permettant d'évaluer l'impact:
    \begin{itemize}
        \item de l'évolution des infrastructures
        \item des changements des règles de régulation
    \end{itemize}
    \pause{}%
    sur la qualité de la mobilité.

    Pour ce faire, nous avons :
    \begin{itemize}
        \item modélisé l'activité de déplacement dans la ville de Cotonou à l'aide des systèmes multi-agents
        \item mis en place le noyau du simulateur pouvant être étendu pour simuler des scénarios de déplacement
    \end{itemize}
\end{frame}


\section{État de l'art}\label{sec:etat_art}
\begin{frame}
    \frametitle{État de l'art}
    \framesubtitle{Amélioration de la mobilité à Cotonou}

    \begin{block}{\citetitle{adonon_problematique}~\autocite{adonon_problematique}}

        Dans~\autocite{adonon_problematique}, une série de solutions potentielles ont été proposées pour améliorer la mobilité à Cotonou.
        Il s'agit principalement de:

        \begin{itemize}
            \item Déconcentrer les activités et densifier l'espace urbain
            \item Mieux organiser l'espace urbain
            \item Atténuer la centralité de la structure des routes
            \item Améliorer les offres de transport publique
        \end{itemize}

    \end{block}
    \pause{}
    Mais les auteurs, n'ont pas pu mesurer les conséquences que peuvent avoir ces décisions sur la mobilité.
\end{frame}

%\begin{frame}
%    \frametitle{État de l'art}
%    \framesubtitle{Simulation des transports}
%%    \begin{block}{Transims~\parencite{smith1995transims} }
%%        \begin{itemize}
%%            \item \emoji{check-mark} Outil de simulation de scénarios de mobilité.
%%            \item \emoji{cross-mark} Ne prends pas en compte les activités économiques autour de la mobilité .
%%        \end{itemize}
%%    \end{block}
%
%\end{frame}

\begin{frame}
    \frametitle{État de l'art}
    \begin{block}{Agent based simulator~\parencite{zargayouna2013agent}}
        \begin{itemize}
            \item \emoji{check-mark} Outil permettant de comprendre et de prédire l'utilisation du réseau routier
%            \item \emoji{cross-mark} Ne modélise pas les activités économiques autour de la mobilité
        \end{itemize}
    \end{block}
    \framesubtitle{Simulation des transports}
    \begin{block}{Plateforme de simulation pour l'aide à la décision~\parencite{nguyen2015plate}}
        \begin{itemize}
            \item \emoji{check-mark} Outil aidant à étudier l'impact des décisions de régulation sur la mobilité
%            \item \emoji{cross-mark} Il ne traite pas de l'évolution des infrastructures
        \end{itemize}
    \end{block}
    \begin{block}{Etude stratégique du réseau routier classé et de la ville de Cotonou~\parencite{louisberger2017etudes}}
        \begin{itemize}
            \item \emoji{check-mark} Modèle permettant d'estimer la demande de mobilité (à partir de: la population, le PIB, le revenu disponible par habitant)
%            \item \emoji{cross-mark} Ne permet pas de comprendre les habitudes de transport
        \end{itemize}
    \end{block}
\end{frame}

\begin{frame}
    \frametitle{État de l'art}
    Les travaux existants ne permettent pas de:
    \framesubtitle{Limites}
    \begin{itemize}
        \item \emoji{cross-mark} Mesurer l'impact de l'évolution des infrastructures
%        \item \emoji{cross-mark} Comprendre les habitudes de transport
        \item \emoji{cross-mark} Prendre en compte les activités économiques autour de la mobilité
        \item \emoji{cross-mark} Prendre en compte de la singularité de la mobilité dans Cotonou
    \end{itemize}

    \pause{}
    La plupart des travaux existant se basent sur les systèmes multi-agents pour modéliser la demande de mobilité et un système qui repose sur deux composantes :
    \begin{itemize}
        \item Un planificateur de route
        \item Et des agents <<\texttt{voyageurs}>> pour représenter les individus
    \end{itemize}
\end{frame}


\section{Modélisation}\label{modelisation}

\begin{frame}
    \frametitle{Modélisation}
    \framesubtitle{Contexte}
    Les transports à Cotonou sont caractérisés par:
    \begin{itemize}
        \item Une forte représentation des engins à 2 roues (entre $60\%$ et $80\%$ selon \textcite{briod2011zemidjan})
        \item Des feux de signalisations défectueux
        \item Des cycles de feux tricolores qui durent longtemps
        \item Des policiers qui, souvent jouent le rôle de régulation local du traffic
        \item Des arrêts brusques de véhicules pour s'approvisionner en carburant
        \item Des stationnements gênants
        \item Des vendeurs ambulants qui profitent des moments d'arrêts des véhicules pour vendre leurs produits
    \end{itemize}

\end{frame}

\begin{frame}
    \frametitle{Modélisation}
    \framesubtitle{Pourquoi un système multi-agents}

    Dans un réseau de transport, un acteur est une entité
    \begin{enumerate}
        \item \textbf{autonome}
        \item qui \textbf{perçoit} l'environnemment qui l'entoure
        \item qui \textbf{interagit} avec les autres acteurs du réseau
        \item capable de \textbf{réagir} aux changements de l'environnement (le réseau routier)
    \end{enumerate}

    \pause{}
    Les acteurs d'un réseau routier sont \textbf{hétérogènes}:

    \begin{itemize}
        \item Utilisent différents types de véhicule
        \item Ont différentes façons de conduire
        \item Réagissent différemment aux changements de l'environmement routier
    \end{itemize}

    \pause{}
    Ces caractéristiques des transports font du système de transport un \textbf{système complexe} et \textbf{constituent le fondement de la modélisation basée sur agent}.
\end{frame}

\begin{frame}
    \frametitle{Modélisation}
    \framesubtitle{Définition d'un système multi-agents}
    Un système multi-agent est un ensemble composé des éléments suivants :
    \begin{itemize}
        \item Un \textbf{environnement}, un espace dans lequel peuvent interagir des acteurs
        \item Un ensemble d'\textbf{objets passifs} ayant une localisation dans l'environnement
        \item Un ensemble d'\textbf{agents} qui sont des objets autonome, capables d'interagir et de manipuler des objets passifs
        \item Un ensemble d'\textbf{opérations} permettant aux agents d'agir sur les autres composants du système
%        \item Un ensemble de \textbf{relations} unissant les objets
    \end{itemize}

    \pause{}
%    Un agent est une entité autonome capable d'agir sur son environnement pour atteindre un objectif.

    Dans le contexte de la modélisation :
    \begin{itemize}
        \item un agent peut être : (vendeur de carburant, conducteur)
        \item un objet peut être : (un bâtiment, une route)
        \item une opération peut être : (conduire, vendre)
    \end{itemize}

\end{frame}

\begin{frame}
    \frametitle{Modélisation}
    \framesubtitle{Modèle de rôles}
    Pour conduire l'analyse de notre recherche, nous utilisons la méthodologie de développement de systèmes multi-agents Gaia.
%    Le modèle de rôles, définit dans la méthodologie de développement de systèmes multi-agents Gaia
    \begin{block}{Modèle de rôles}
        \begin{itemize}
            \item Modèle abstrait
            \item Représente les différentes fonctions que doivent exposer les agents de notre système
        \end{itemize}
    \end{block}

    \pause{}
    C'est ainsi que nous avons défini les rôles :

    \begin{itemize}
        \item \rConducteur{}: Individus ayant un moyen de déplacement et qui doivent se déplacer
        \item \rTaxi{}: Individus offrant des services de <<taxi>>
        \item \rPieton{}: Personnes voulant se déplacer utilisant le service d'un \rTaxi{}
%        \item \rRegulateurTrafic{}: Personnes assurant la régulation du traffic
        \item \rPolicier{}: Policiers assurant une partie de la régulation du traffic
        \item \rVCarburant{}: Vendeur de carburant
        \item \rGps{}: Personnes qui aident les \rConducteur{} à s'orienter
        \item \rPertubateur{}: Personnes qui troublent le traffic à cause de leurs activités
    \end{itemize}

%    ../../0-final-report/src/4/models/role-interaction/role-graph.tex
\end{frame}

%\begin{frame}[shrink=20]
%    \frametitle{Modélisation}
%    \framesubtitle{Description du rôle \rConducteur{}}
%
%    \begin{center}
%        \roleTable{\rConducteur{}}
%        {Le \rConducteur{} représente une personne qui mène des activités de déplacement dans la vie réelle.
%        Il a des activités quotidiennes à mener.}
%        {\behAskPath{},  \behNegotiateFuel{}, \behAskParking{}, \behIssueStats{}}
%        {\behAlwaysMove{}, \behSearchFuelStation{},  \behPark{}, \behCollectLocalStats{}}
%        {\permsHaveActivities{}, \permsUseFuel{}, \permsUseRoads{}}
%        {
%        $
%        \rConducteur{}=(
%        \behAlwaysMove{}.
%        (
%        \behSearchFuelStation{}.
%        \behNegotiateFuel{}
%        )^{*}. \allowbreak
%        \behAskPath{}^{*}.
%        (
%        \behAskParking{}.
%        \behPark{}
%        )^{*}. \allowbreak
%        \behCollectLocalStats{}. \allowbreak
%        \behIssueStats{}
%        )^{\omega}
%        $
%        }
%        {\safetyMaxActivityLimit{}, \safetyCanMoveFuelEmpty{}, \safetyBuyIfNoMoney{}}
%        {roleconducteur}
%
%    \end{center}
%\end{frame}

%\begin{frame}[shrink=40, fragile]
%    \frametitle{Modélisation}
%    \framesubtitle{Description du rôle \rConducteur{}}
%
%    \begin{center}
%        \protocoleTable{\behAskPath{}}
%        {\rConducteur{}}
%        {\rGps{}}
%        {\varCurrentLocation{}, \varDestination{}}
%        {Demander un chemin vers le lieu où le \rConducteur{} doit se diriger à un rôle \rGps{}}
%        {\texttt{List<\varPaths{}>}}
%        {\rConducteur{}}{protocole-ask-path-role-conducteur}
%
%        \protocoleTable{\behAskParking{}}
%        {\rConducteur{}}
%        {\rGps{}}
%        {\varCurrentLocation{}, \varVehiculeType{}}
%        {Demander un endroit où le \rConducteur{} peut garer son véhicule au rôle \rGps{}}
%        {\texttt{List<\varParkings{}>}}
%        {\rConducteur{}}{protocole-ask-parking-role-conducteur}
%
%        \protocoleTable{\behNegociateFuelPrice{}}
%        {\rConducteur{}}
%        {\rVCarburant{}}
%        {\varFuelQuantity{}}
%        {Acheter du carburant chez un \rVCarburant{}}
%        {\varFuel{} en cas de réussite, le \rConducteur{} passe à un autre \rVCarburant{}}
%        {\rConducteur{}}{protocole-provide-fuel-role-conducteur}
%
%    \end{center}
%
%\end{frame}


%\begin{frame}[shrink=20]
%    \frametitle{Modélisation}
%    \framesubtitle{Description du rôle \rInitialisateur{}}
%
%
%    \roleTable{\rGenerateurPop{}}
%    {Générer les populations de notre système à partir de données réelles}
%    {}
%    {\behAddConducteur{}}
%    {\permsMobilityData{}, \permsAddAgent{}}
%    {$\rGenerateurPop{}=(\behAddConducteur{})^{*}$}
%    {\safetyPopLimit{}}
%    {role-generateur}
%
%\end{frame}

\begin{frame}[shrink=25]
    \frametitle{Modélisation}
    \framesubtitle{Modèle d'agents}
    Le modèle d'agent est dérivé du modèle de rôle et permet de représenter les différentes implémentations des rôles du système.
    \begin{figure}[!htbp]
        \centering
        \begin{tikzpicture}
            \begin{scope}[every node/.style={rectangle,thick,draw}]
                \node[draw=black, label={-75:+}] (agentConducteur) at (0,2) {\aConducteur{}};
                \node[draw=red] (roleConducteur)  at (0,0) {\rConducteur{}};

                \node[draw=black, label={-75:+}] (aTVoiture) at (4,2) {\aTVoiture{}};
                \node[draw=black, label={-75:+}] (aTMoto) at (8,2) {\aTMoto{}};
                \node[draw=black, label={-75:+}] (aTBus) at (12,2) {\aTBus{}};
                \node[draw=red] (rTaxi)  at (8,0) {\rTaxi{}};

                \node[draw=black, label={-87:+}] (aPieton) at (0,-2) {\aPieton{}};
                \node[draw=black, label={-75:+}] (aVAmbulant) at (4,-2) {\aVAmbulant{}};
                \node[draw=red] (rPieton)  at (0,-4) {\rPieton{}};
                \node[draw=red] (rGps)  at (2,-4) {\rGps{}};
                \node[draw=red] (rPertubateur)  at (4,-4) {\rPertubateur{}};

                \node[draw=black, label={-75:+}] (aPolicier)  at (8,-2) {\aPolicier{}};
                \node[draw=black, label={-75:+}] (aRegulateur)  at (12,-2) {\aRegulateurTrafic{}};
                \node[draw=red] (rPolicier)  at (8,-4) {\rPolicier{}};
                \node[draw=red] (rRegulateur)  at (12,-4) {\rRegulateurTrafic{}};


            \end{scope}

            \begin{scope}[>={Stealth[black]}, every node/.style={fill=white,rectangle}, every edge/.style={draw=black}]
                \path [<-] (agentConducteur) edge (roleConducteur);
                \path [<-] (aTVoiture) edge (rTaxi);
                \path [<-] (aTMoto) edge (rTaxi);
                \path [<-] (aTBus) edge (rTaxi);
                \path [<-] (aPieton) edge (rPieton);
                \path [<-] (aPieton) edge (rPertubateur);
                \path [<-] (aPieton) edge (rGps);
                \path [<-] (aVAmbulant) edge (rPertubateur);
                \path [<-] (aPolicier) edge (rPolicier);
                \path [<-] (aPolicier) edge (rRegulateur);
                \path [<-] (aRegulateur) edge (rRegulateur);

            \end{scope}
        \end{tikzpicture}
        Les rôles sont bordés d'un rectangle rouge tandis que les types d'agent sont bordés d'un rectangle noir.
        \caption{Modèle d'agent}
        \label{fig:agent-model}
    \end{figure}
\end{frame}
%
%\begin{frame}[shrink]
%    \frametitle{Modélisation}
%    \framesubtitle{Modèles concrets}
%    \begin{figure}[!htbp]
%        \centering
%        \begin{tikzpicture}
%            \begin{scope}[every node/.style={rectangle,thick,draw}]
%
%
%                \node[draw=black, label={-90:+}] (aVIllicite)  at (0,-6) {\aVIllicite{}};
%                \node[draw=black, label={-75:+}] (aVStation)  at (4,-6) {\aVStation{}};
%                \node[draw=red] (rVCarburant)  at (2,-8) {\rVCarburant{}};
%
%                \node[draw=black, label={-75:1}] (aGPopulation)  at (9,-6) {\aGPopulation{}};
%                \node[draw=red] (rGenerateurPop)  at (9,-8) {\rGenerateurPop{}};
%
%                \node[draw=black, label={-75:1}] (aAssocActivite)  at (2,-10) {\aAssocActivite{}};
%                \node[draw=red] (rAssociateurActivite)  at (2,-12) {\rAssociateurActivite{}};
%
%                \node[draw=black, label={-75:1}] (aStatistiques)  at (9,-10) {\aStats{}};
%                \node[draw=red] (rStatistiques)  at (9,-12) {\rStatistiques{}};
%
%            \end{scope}
%
%            \begin{scope}[>={Stealth[black]}, every node/.style={fill=white,rectangle}, every edge/.style={draw=black}]
%                \path [<-] (aVStation) edge (rVCarburant);
%                \path [<-] (aVIllicite) edge (rVCarburant);
%                \path [<-] (aGPopulation) edge (rGenerateurPop);
%
%                \path [<-] (aAssocActivite) edge (rAssociateurActivite);
%                \path [<-] (aStatistiques) edge (rStatistiques);
%
%            \end{scope}
%        \end{tikzpicture}
%        Les rôles sont bordés d'un rectangle rouge tandis que les types d'agent sont bordés d'un rectangle noir.
%        \caption{Modèle d'agent}
%        \label{fig:agent-model}
%    \end{figure}
%
%
%\end{frame}

\begin{frame}
    \frametitle{Modélisation}
    \framesubtitle{Méta-rôles}

    Nous avons aussi besoin de certaines fonctions qui ne peuvent pas être représentées avec les rôles précédents.
    \begin{block}{Méta-rôles}
        \begin{itemize}
%            \item \rIndReel{}: Représente les tous les rôles mettant en action des individus dans la simulation
            \item \rGenerateurPop{}: Crée les agents à partir des données caractérisant les transports à \ctn{}:
            \begin{itemize}
                \item le nombre de taxi moto
                \item le nombre de conducteurs simple
                \item les habitudes de conduite des conducteurs
                \item le nombre de piétons
            \end{itemize}
            \item \rAssociateurActivite{}: Permet d'associer à chaque agent du système un ensemble de déplacements à effectuer:
            \begin{itemize}
                \item aller au travail
                \item faire des courses
                \item se déplacer vers un lieu
            \end{itemize}
        \end{itemize}
    \end{block}
\end{frame}


\section{Implémentation}\label{implementation}
\begin{frame}
    \frametitle{Implémentation}
    \framesubtitle{Architecture}

    \begin{block}{Simulateur GAMA}
        Pour mettre en place notre simulation, nous avons utilisé le simulateur GAMA:
        \begin{itemize}
            \item Une plateforme permettant de créer des simulateurs basés sur la technologie multi-agents
            \item Conçu particulièrement pour manipuler les données spatiales
        \end{itemize}

    \end{block}

\end{frame}


\begin{frame}[shrink]
    \frametitle{Implémentation}
    \framesubtitle{Architecture}
    Notre système est composé de 2 modules:

    \begin{columns}[c]
        \begin{column}{0.45\textwidth}
            \begin{itemize}
                \item Module d'interface
                \begin{itemize}
                    \item Affichage
                    \item Configuration de paramètres
                \end{itemize}
                \item C\oe ur du simulateur
                \begin{itemize}
                    \item Chargeur de configuration
                    \item Espèces d'agents
                    \item Unité de traitement
                \end{itemize}
            \end{itemize}

        \end{column}
        \pause{}%
        \begin{column}{0.45\textwidth}
            \begin{figure}[h]
                \centering
                \includegraphics[width=0.6\textwidth]{architecture_simulateur.pdf}
                \caption{Architecture de notre système}
                \label{fig:architecture-simulateur}
            \end{figure}
        \end{column}
    \end{columns}

\end{frame}

\begin{frame}
    \frametitle{Implémentation}
    \framesubtitle{Architecture}
    \textbf{Chargeur de configuration}
    \begin{itemize}
        \item charge les données (environnement) nécessaires pour la simulation. (Ex. un fichier de format <<shapefile>> décrivant les routes et les bâtiments)

    \end{itemize}

    \textbf{Espèces d'agent}
    \begin{itemize}
        \item contient la définition des agents du système (Ex. \aTMoto{}, \aTBus{}, \aPolicier{})
    \end{itemize}

    \textbf{Unité de traitement}
    \begin{itemize}
        \item construit l'environnement
        \item charge les agents définis dans \texttt{Espèces d'agents}
        \item ajoute les objets
    \end{itemize}

\end{frame}


\section{Résultats}\label{results}
\begin{frame}
    \frametitle{Résultats}
    \framesubtitle{Scénario simulé}

    \begin{block}{Scénario}
        Pour tester le fonctionnement de notre modèle, nous avons représenté à l'aide de notre simulateur un scénario dans lequel nous représentons le trajet domicile-travail des résidents de Cotonou.
        Un \aConducteur{} dans notre scénario
        \begin{itemize}
            \item Commence une journée à 6 h et termine à 21 h
            \item Se déplace pour aller au travail et ensuite rentrer à son domicile
        \end{itemize}
        Ce scénario nous permet de représenter une activité qui constitue une part importante des motifs de déplacements dans la ville de Cotonou.
    \end{block}

\end{frame}

\begin{frame}
    \frametitle{Résultats}
    \framesubtitle{Données d'entrée}

    \begin{block}{Données d'entrée}
        Pour mettre en place l'environnement de la simulation, notre simulateur prend en entrée :
        \begin{itemize}
            \item L'ensemble des routes
            \item L'ensemble des bâtiments
            \item Les informations démographiques (zones habitées, le nombre d'habitants)
        \end{itemize}

        Les informations des routes, les bâtiments sont des fichiers de format <<shapefile>> que nous avons collecté auprès de l'IGN et du Ministère des infrastructures et des transports.

%        \begin{itemize}
%            \item
%        \end{itemize}
    \end{block}

\end{frame}

\begin{frame}
    \frametitle{Résultats}
    \framesubtitle{Vue sur notre outil de simulation}

    \begin{figure}[h]
        \centering
        \includegraphics[width=\textwidth]{../0-final-report/src/5/img/overview_simulateur.png}
        \caption{Vue sur notre outil de simulation}
        \label{fig:simulateur-overview}
    \end{figure}

\end{frame}

\begin{frame}
    \frametitle{Résultats}
    \framesubtitle{Vue sur notre outil de simulation}

    Nous pouvons remarquer certains lieux caractéristiques de la ville de \ctn{}:
    \begin{itemize}
        \item La place de l'étoile rouge
        \item Le marché Dantokpa
        \item Le carrefour Vèdoko
        \item Le stade de l'Amitié
    \end{itemize}
    \begin{figure}[h]
        \centering
        \includegraphics[width=0.9\textwidth]{../0-final-report/src/5/img/ctn_annote.pdf}
        \caption{Annotation de certaines infrastructures de \ctn{}}
        \label{fig:ctn-anotated}
    \end{figure}

\end{frame}


\begin{frame}
    \frametitle{Résultats}
    \framesubtitle{Démonstration}

    \begin{center}
        Démonstration
    \end{center}
\end{frame}

\begin{frame}
    \frametitle{Résultats}
    \framesubtitle{Vue sur le déroulement de la simulation}

    \begin{figure}[H]
        \centering
        \includegraphics[width=0.7\textwidth]{../0-final-report/src/5/img/sim-running.png}
        \caption{Vue sur le déroulement de la simulation}
        \label{fig:simulateur-running}
    \end{figure}
    \begin{itemize}
        \item Les points bleus représentent des agents en déplacement.
        \item Les points rouges représentent des lieux de travail.
    \end{itemize}
\end{frame}

\begin{frame}
    \frametitle{Résultats}
    \framesubtitle{Routes les plus utilisées pour le scénario simulé}

    \begin{figure}[h]
        \centering
        \includegraphics[width=\textwidth]{../0-final-report/src/5/img/Figure_1.png}
        \caption{Routes les plus utilisées pour le scénario simulé}
        \label{fig:top_used_road}
    \end{figure}

\end{frame}

\begin{frame}
    \frametitle{Résultats}
    \framesubtitle{Utilisation des routes suivant l'heure}

    \begin{figure}[h]
        \centering
        \includegraphics[width=\textwidth]{../0-final-report/src/5/img/Figure_2.png}
        \caption{Utilisation des routes suivant l'heure}
        \label{fig:road_usage_hour}
    \end{figure}

\end{frame}

\begin{frame}
    \frametitle{Résultats}
    \framesubtitle{Interprétation}

    \begin{block}{Interprétation}
        Avec de tels graphiques, les décideurs peuvent:
        \begin{itemize}
            \item connaître les infrastructures qui sont les plus sollicitées pour le scénario simulé.
            \item observer les comportements des conducteurs du scénario
        \end{itemize}

        Ils peuvent changer les données des routes en entrée pour:
        \begin{itemize}
            \item représenter un changement d'infrastructure
            \item voir le nouveau comportement des conducteurs face au changement
        \end{itemize}
    \end{block}
\end{frame}

\section{Conclusion}
\begin{frame}
    \frametitle{Conclusion}
    Nous avons construit une base d'outil de simulation permettant de:
    \begin{itemize}
        \item \emoji{check-mark} Mesurer l'impact de l'évolution des infrastructures
        \item \emoji{check-mark} Comprendre des habitudes de transport
        \item \emoji{check-mark} Prendre en compte les activités économiques autour de la mobilité
        \item \emoji{check-mark} Prendre en compte de la singularité des déplacements dans Cotonou
    \end{itemize}


    \begin{alertblock}{Limites}
        \begin{itemize}
            \item Non prise en compte de la qualité des routes
            \item Le simulateur est gourmand en ressources (CPU, RAM)
        \end{itemize}
    \end{alertblock}

    \begin{block}{Perspectives}
        \begin{itemize}
            \item Implémenter les scénarios qui sont dans les projets d'urbanisation du Ministère des infrastructures et des transports (Contournement nord de Cotonou)
            \item Implémenter des scénarios de mobilité plus complexe
        \end{itemize}
    \end{block}
\end{frame}

%   On observe une efferverce
%    \begin{itemize}
%        \item Elasticsearch est un moteur de recherche basé sur la bibliothèque \href{https://en.wikipedia.org/wiki/Lucene}{Lucene}\autocite{wiki_elastic}.
%        \item Il fournit un moteur de recherche "full text" distribué, avec une interface web HTTP (api rest) et des documents JSON non structurées\autocite{wiki_elastic}.
%    \end{itemize}
%
%    \begin{block}{Régulation de traffic}
%    \end{block}

%    \begin{block}{Régulation de traffic}
%        Des systèmes de régulation existents pour assurer une bonne qualité de la mobilité.
%    \end{block}
