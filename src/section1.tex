\section{Introduction}\label{sec:section-1}
\begin{frame}
    \frametitle{Contexte}
    Selon le rapport~\autocite{somik2017opening}, la population des villes africaines est en forte croissance.

    \begin{itemize}
        \item La forte occupation des villes augmente leur dynamisme, la demande de mobilité et influe sur leur développement
        \item Un autre aspect influant le développement d'une ville est la qualité de la mobilité des personnes et des biens relativement à la ville.
        \item La fluidité de la mobilité est impactée par les règles de régulation du traffic routier et par les changements au niveau des infrastructures routières.
    \end{itemize}

\end{frame}

\begin{frame}
    \frametitle{Problématique}

    \begin{itemize}
        \item Les infrastructures routières et les outils de régulation du traffic doivent évoluer pour supporter la demande croissante de mobilité.
        \item Pour prendre des décisions sur l'évolution de ces infrastructures, il faut pouvoir évaluer l'impact:
        \begin{itemize}
            \item de l'évolution des infrastructures routières sur la qualité de la mobilité
            \item des changements des règles de régulation routière sur la qualité de la mobilité
        \end{itemize}
    \end{itemize}
    \begin{alertblock}{Problème}

        \begin{itemize}
            \item C'est un travail fastidieux qui peut rapidement se complexifier
            \item Ce n'est pas évident de le faire manuellement
        \end{itemize}
    \end{alertblock}
\end{frame}

\begin{frame}
    \frametitle{Objectifs}
    Mettre en place un simulateur permettant d'évaluer :
    \begin{itemize}
        \item l'impact de l'évolution des infrastructures sur la qualité de la mobilité
        \item des changements des règles de régulation sur la qualité de la mobilité
    \end{itemize}
    \pause{}%
    Pour ce faire, nous avons :
    \begin{itemize}
        \item modélisé le flux de mobilité à l'aide des systèmes multi-agents en nous basant sur la ville de Cotonou
        \item mis en place le noyau du simulateur pouvant être étendu pour simuler des scénarios de mobilité
    \end{itemize}
\end{frame}


\section{État de l'art}\label{sec:etat_art}
\begin{frame}
    \frametitle{État de l'art}
    \framesubtitle{Amélioration de la mobilité à Cotonou}

    Dans~\autocite{adonon_problematique}, une série de solutions potentielles ont été porposées pour améliorer la mobilité à Cotonou.
    Il s'agit principalement de

    \begin{itemize}
        \item Atténuer la centralité de la structure des routes
        \item Déconcentrer les activités et densifier l'espace urbain
        \item Améliorer les offres de transport publique
        \item Mieux organiser l'espace urbain
    \end{itemize}
\end{frame}

\begin{frame}
    \frametitle{État de l'art}
    \framesubtitle{Simulation des transports}
    \begin{block}{Transims~\parencite{smith1995transims} }
        \begin{itemize}
            \item \emoji{check-mark} Outil de simulation de scénarios de mobilité.
            \item \emoji{cross-mark} Ne prends pas en compte les activités économiques autour de la mobilité .
        \end{itemize}
    \end{block}
    \begin{block}{Agent based simulator~\parencite{zargayouna2013agent}}
        \begin{itemize}
            \item \emoji{check-mark} Outil permettant de comprendre et de prédire l'utilisation du réseau routier
            \item \emoji{cross-mark} Ne modélise pas les activités économiques autour de la mobilité
        \end{itemize}
    \end{block}
\end{frame}

\begin{frame}
    \frametitle{État de l'art}
    \framesubtitle{Simulation des transports}
    \begin{block}{Plateforme de simulation pour l'aide à la décision~\parencite{nguyen2015plate}}
        \begin{itemize}
            \item \emoji{check-mark} Outil aidant à étudier l'impact des décisions de régulation sur la mobilité
            \item \emoji{cross-mark} Il ne traite pas de l'évolution des infrastructures
        \end{itemize}
    \end{block}
    \begin{block}{Etude stratégique du réseau routier classé et de la ville de Cotonou~\parencite{louisberger2017etudes}}
        \begin{itemize}
            \item \emoji{check-mark} Modèle permettant d'estimer la demande de mobilité (à partir de: )
            \begin{itemize}
                \item population,
                \item pib,
                \item revenu disponible par habitant
            \end{itemize}
            \item \emoji{cross-mark} Ne permets pas de comprendre les habitudes de transport
        \end{itemize}
    \end{block}
\end{frame}

\begin{frame}
    \frametitle{État de l'art}
    Les travaux existant ont des limites
    \framesubtitle{Limites}
    \begin{itemize}
        \item \emoji{cross-mark} Mesure de l'impact de l'évolution des infrastructures
        \item \emoji{cross-mark} Compréhension des habitudes de transport
        \item \emoji{cross-mark} Prise en compte les activités économiques autour de la mobilité
        \item \emoji{cross-mark} Prise en compte de la singularité de la mobilité dans Cotonou
    \end{itemize}
    \pause{}
    La plupart des travaux existant se basent sur les systèmes multi-agents pour modéliser la demande de mobilité et
    utilisent l'approche orienté sur les activités pour représenter les motifs de déplacements.
    Ils utilisent aussi une structure qui repose sur deux composantes :
    \begin{itemize}
        \item Un planificateur de route
        \item Et des agents <<\texttt{voyageurs}>> pour représenter les individus.
    \end{itemize}
\end{frame}


\section{Modélisation}\label{modelisation}

\begin{frame}
    \frametitle{Modélisation}
    \framesubtitle{Contexte}
    La mobilité à Cotonou est caractérisée par:
    \begin{itemize}
        \item Une forte présence des engins à 2 roues (entre $60\%$ et $80\%$ selon \textcite{briod2011zemidjan})
        \item Des feux de signalisations défectueux
        \item Des cycles de feux qui durent longtemps
        \item Des policiers qui souvent le rôle de régulation local du traffic
        \item Des arrêts brusques de véhicules pour s'approvisionner en carburant
        \item Des stationnements encombrant
        \item Des vendeurs ambulants qui profitent des moments d'arrêts des véhicules pour vendre leurs produit
    \end{itemize}

\end{frame}

\begin{frame}
    \frametitle{Modélisation}
    \framesubtitle{Pourquoi un système multi-agents}

    Dans un réseau de transport, un acteur est une entité
    \begin{enumerate}
        \item \textbf{autonome}
        \item qui \textbf{perçoit} l'environnemment qui l'entoure
        \item qui \textbf{interagit} avec les autres acteurs du réseau
        \item capable de \textbf{réagir} aux changements de l'environnement (le réseau routier)
    \end{enumerate}

    \pause{}
    Les acteurs d'un réseau routier sont \textbf{hétérogène}:

    \begin{itemize}
        \item Utilisent différents types de véhicule
        \item Ont différentes façons de conduire
        \item Réagissent différemment aux changements de l'environmement routier
    \end{itemize}

    \pause{}
    Ces caractéristiques du réseau routier font du système de transport un \textbf{système complexe} et \textbf{constituent le fondement de la modélisation basée sur agent}.
\end{frame}

\begin{frame}
    \framesubtitle{Modélisation}
    \frametitle{Définition d'un système multi-agents}
    Un système multi-agent est un ensemble composé des éléments suivant:
    \begin{itemize}
        \item Un \textbf{environnement}, un espace dans lequel peuvent interagir des acteurs
        \item Un ensemble d'\textbf{objets passifs} situés
        \item Un ensemble d'\textbf{agents} qui sont des objects capable d'interagir et de manipuler des object passifs
        \item Un ensemble d'\textbf{opération} permettant aux agents d'agir sur les autres composant du système
        \item Un ensemble de \textbf{relations} unissant les objets
    \end{itemize}

    \pause{}
    Un agent est une entité autonome capable d'agir sur son environnement pour atteindre un objectif.

    Dans le contexte de la modélisation, il s'agit par exemple d'un conducteur, ou d'un vendeur de carburant, ou d'une personne ayant comme objectif de se déplacer.
\end{frame}

\begin{frame}[shrink]
    \frametitle{Modélisation}
    \framesubtitle{Modèles de rôle}
    Pour conduire l'analyse de notre recherche, nous utilisons la méthodologie de développement de systèmes multi-agents Gaia.
    Elle nous amène à définir un modèle abstrait des différents rôles présent dans notre système multi-agents:

    ../../0-final-report/src/4/models/role-interaction/role-graph.tex
\end{frame}

\begin{frame}[shrink=20]
    \frametitle{Modélisation}
    \framesubtitle{Description du rôle \rConducteur{}}

    \begin{center}
        \roleTable{\rConducteur{}}
        {Le \rConducteur{} représente une personne qui mène des activités de déplacement dans la vie réelle.
        Il a des activités quotidiennes à mener.}
        {\behAskPath{},  \behNegotiateFuel{}, \behAskParking{}, \behIssueStats{}}
        {\behAlwaysMove{}, \behSearchFuelStation{},  \behPark{}, \behCollectLocalStats{}}
        {\permsHaveActivities{}, \permsUseFuel{}, \permsUseRoads{}}
        {
        $
        \rConducteur{}=(
        \behAlwaysMove{}.
        (
        \behSearchFuelStation{}.
        \behNegotiateFuel{}
        )^{*}. \allowbreak
        \behAskPath{}^{*}.
        (
        \behAskParking{}.
        \behPark{}
        )^{*}. \allowbreak
        \behCollectLocalStats{}. \allowbreak
        \behIssueStats{}
        )^{\omega}
        $
        }
        {\safetyMaxActivityLimit{}, \safetyCanMoveFuelEmpty{}, \safetyBuyIfNoMoney{}}
        {roleconducteur}

    \end{center}
\end{frame}

\begin{frame}[shrink=40, fragile]
    \frametitle{Modélisation}
    \framesubtitle{Description du rôle \rConducteur{}}

    \begin{center}
        \protocoleTable{\behAskPath{}}
        {\rConducteur{}}
        {\rGps{}}
        {\varCurrentLocation{}, \varDestination{}}
        {Demander un chemin vers le lieu où le \rConducteur{} doit se diriger à un rôle \rGps{}}
        {\texttt{List<\varPaths{}>}}
        {\rConducteur{}}{protocole-ask-path-role-conducteur}

        \protocoleTable{\behAskParking{}}
        {\rConducteur{}}
        {\rGps{}}
        {\varCurrentLocation{}, \varVehiculeType{}}
        {Demander un endroit où le \rConducteur{} peut garer son véhicule au rôle \rGps{}}
        {\texttt{List<\varParkings{}>}}
        {\rConducteur{}}{protocole-ask-parking-role-conducteur}

        \protocoleTable{\behNegociateFuelPrice{}}
        {\rConducteur{}}
        {\rVCarburant{}}
        {\varFuelQuantity{}}
        {Acheter du carburant chez un \rVCarburant{}}
        {\varFuel{} en cas de réussite, le \rConducteur{} passe à un autre \rVCarburant{}}
        {\rConducteur{}}{protocole-provide-fuel-role-conducteur}

    \end{center}

\end{frame}


\begin{frame}[shrink=20]
    \frametitle{Modélisation}
    \framesubtitle{Description du rôle \rInitialisateur{}}


    \roleTable{\rGenerateurPop{}}
    {Générer les populations de notre système à partir de données réelles}
    {}
    {\behAddConducteur{}}
    {\permsMobilityData{}, \permsAddAgent{}}
    {$\rGenerateurPop{}=(\behAddConducteur{})^{*}$}
    {\safetyPopLimit{}}
    {role-generateur}

\end{frame}

\begin{frame}[shrink=25]
    \frametitle{Modélisation}
    \framesubtitle{Modèles concrets}
    Des modèles abstraits définit plus tôt, nous devons extraire des modèles concrets d'agent.
    \begin{figure}[!htbp]
        \centering
        \begin{tikzpicture}
            \begin{scope}[every node/.style={rectangle,thick,draw}]
                \node[draw=black, label={-75:+}] (agentConducteur) at (0,2) {\aConducteur{}};
                \node[draw=red] (roleConducteur)  at (0,0) {\rConducteur{}};

                \node[draw=black, label={-75:+}] (aTVoiture) at (4,2) {\aTVoiture{}};
                \node[draw=black, label={-75:+}] (aTMoto) at (8,2) {\aTMoto{}};
                \node[draw=black, label={-75:+}] (aTBus) at (12,2) {\aTBus{}};
                \node[draw=red] (rTaxi)  at (8,0) {\rTaxi{}};

                \node[draw=black, label={-87:+}] (aPieton) at (0,-2) {\aPieton{}};
                \node[draw=black, label={-75:+}] (aVAmbulant) at (4,-2) {\aVAmbulant{}};
                \node[draw=red] (rPieton)  at (0,-4) {\rPieton{}};
                \node[draw=red] (rGps)  at (2,-4) {\rGps{}};
                \node[draw=red] (rPertubateur)  at (4,-4) {\rPertubateur{}};

                \node[draw=black, label={-75:+}] (aPolicier)  at (8,-2) {\aPolicier{}};
                \node[draw=black, label={-75:+}] (aRegulateur)  at (12,-2) {\aRegulateurTrafic{}};
                \node[draw=red] (rPolicier)  at (8,-4) {\rPolicier{}};
                \node[draw=red] (rRegulateur)  at (12,-4) {\rRegulateurTrafic{}};


            \end{scope}

            \begin{scope}[>={Stealth[black]}, every node/.style={fill=white,rectangle}, every edge/.style={draw=black}]
                \path [<-] (agentConducteur) edge (roleConducteur);
                \path [<-] (aTVoiture) edge (rTaxi);
                \path [<-] (aTMoto) edge (rTaxi);
                \path [<-] (aTBus) edge (rTaxi);
                \path [<-] (aPieton) edge (rPieton);
                \path [<-] (aPieton) edge (rPertubateur);
                \path [<-] (aPieton) edge (rGps);
                \path [<-] (aVAmbulant) edge (rPertubateur);
                \path [<-] (aPolicier) edge (rPolicier);
                \path [<-] (aPolicier) edge (rRegulateur);
                \path [<-] (aRegulateur) edge (rRegulateur);

            \end{scope}
        \end{tikzpicture}
        Les rôles sont bordés d'un rectangle rouge tandis que les types d'agent sont bordés d'un rectangle noir.
        \caption{Modèle d'agent}
        \label{fig:agent-model}
    \end{figure}
\end{frame}
\begin{frame}[shrink]
    \frametitle{Modélisation}
    \framesubtitle{Modèles concrets}
    \begin{figure}[!htbp]
        \centering
        \begin{tikzpicture}
            \begin{scope}[every node/.style={rectangle,thick,draw}]


                \node[draw=black, label={-90:+}] (aVIllicite)  at (0,-6) {\aVIllicite{}};
                \node[draw=black, label={-75:+}] (aVStation)  at (4,-6) {\aVStation{}};
                \node[draw=red] (rVCarburant)  at (2,-8) {\rVCarburant{}};

                \node[draw=black, label={-75:1}] (aGPopulation)  at (9,-6) {\aGPopulation{}};
                \node[draw=red] (rGenerateurPop)  at (9,-8) {\rGenerateurPop{}};

                \node[draw=black, label={-75:1}] (aAssocActivite)  at (2,-10) {\aAssocActivite{}};
                \node[draw=red] (rAssociateurActivite)  at (2,-12) {\rAssociateurActivite{}};

                \node[draw=black, label={-75:1}] (aStatistiques)  at (9,-10) {\aStats{}};
                \node[draw=red] (rStatistiques)  at (9,-12) {\rStatistiques{}};

            \end{scope}

            \begin{scope}[>={Stealth[black]}, every node/.style={fill=white,rectangle}, every edge/.style={draw=black}]
                \path [<-] (aVStation) edge (rVCarburant);
                \path [<-] (aVIllicite) edge (rVCarburant);
                \path [<-] (aGPopulation) edge (rGenerateurPop);

                \path [<-] (aAssocActivite) edge (rAssociateurActivite);
                \path [<-] (aStatistiques) edge (rStatistiques);

            \end{scope}
        \end{tikzpicture}
        Les rôles sont bordés d'un rectangle rouge tandis que les types d'agent sont bordés d'un rectangle noir.
        \caption{Modèle d'agent}
        \label{fig:agent-model}
    \end{figure}


\end{frame}


\section{Implémentation}\label{implementation}
\begin{frame}
    \frametitle{Implémentation}
    \framesubtitle{Architecture}

    Pour mettre en place notre simulation, nous avons utilisé le simulateur GAMA:
    \begin{itemize}
        \item Une plateforme permettant de créer des simulateurs basés sur la technologie multi-agents
        \item Conçu particulièrement pour manipuler les données spatiales .
    \end{itemize}

    Notre système est composé de 2 module:
    \begin{itemize}
        \item Module d'interface
        \begin{itemize}
            \item Affichage
            \item Configuration de paramètres
        \end{itemize}
        \item C\oe ur du simulateur
        \begin{itemize}
            \item Chargeur de configuration
            \item Espèces d'agent
            \item Unité de traitement
        \end{itemize}
    \end{itemize}

\end{frame}


\begin{frame}
    \frametitle{Implémentation}
    \framesubtitle{Architecture}

    \begin{figure}[h]
        \centering
        \includegraphics[width=0.3\textwidth]{architecture_simulateur.pdf}
        \caption{Architecture de notre système}
        \label{fig:architecture-simulateur}
    \end{figure}
\end{frame}

\begin{frame}
    \frametitle{Implémentation}
    \framesubtitle{Architecture}
    \textbf{Chargeur de configuration}
    \begin{itemize}
        \item charge les données (environnement) nécessaires pour la simulation. (Ex. un fichier shapefile décrivant les routes et les bâtiments)

    \end{itemize}

    \textbf{Espèces d'agent}
    \begin{itemize}
        \item contient la définition des agents du système (Ex. \aTMoto{}, \aTBus{}, \aPolicier{})
    \end{itemize}

    \textbf{Unité de traitement}
    \begin{itemize}
        \item construit l'environnement
        \item charge les agents définit dans \texttt{Espèces d'agent}
        \item ajoute les objets (artéfacts)
    \end{itemize}

\end{frame}


\section{Résultats}\label{results}
\begin{frame}
    \frametitle{Résultat}
    \framesubtitle{Scénario simulé}

    \begin{block}{Scénario}
        Pour tester le fonctionnement de notre modèle, nous avons représenté à l'aide de notre simulateur un scénario dans lequel nous représentons l'aller au travail des résidents de Cotonou.
        Un agent conducteur dans notre scénario
        \begin{itemize}
            \item Commence une journée à 6h et se terminant à 21h
            \item Se déplace pour aller au boulot et ensuite rentré à son domicile
        \end{itemize}
        Ce scénario nous permet de représenter une activité qui constitue une part importante des motifs de déplacements dans la ville de Cotonou.
    \end{block}
\end{frame}

\begin{frame}
    \frametitle{Résultat}
    \framesubtitle{Vue sur notre outil de simulation}

    \begin{figure}[h]
        \centering
        \includegraphics[width=\textwidth]{../0-final-report/src/5/img/overview_simulateur.png}
        \caption{Vue sur notre outil de simulation}
        \label{fig:simulateur-overview}
    \end{figure}


\end{frame}

\begin{frame}
    \frametitle{Résultat}
    \framesubtitle{Vue sur notre outil de simulation}

    \begin{figure}[h]
        \centering
        \includegraphics[width=\textwidth]{../0-final-report/src/5/img/ctn_annote.pdf}
        \caption{Annotation de certaines infrastructures de \ctn{}}
        \label{fig:ctn-anotated}
    \end{figure}

\end{frame}

\begin{frame}
    \frametitle{Résultat}
    \framesubtitle{Routes les plus utilisées pour le scénario simulé}

    \begin{figure}[h]
        \centering
        \includegraphics[width=\textwidth]{../0-final-report/src/5/img/Figure_1.png}
        \caption{Routes les plus utilisées pour le scénario simulé}
        \label{fig:top_used_road}
    \end{figure}

\end{frame}
\begin{frame}
    \frametitle{Résultat}
    \framesubtitle{Utilisation des routes suivant l'heure}

    \begin{figure}[h]
        \centering
        \includegraphics[width=\textwidth]{../0-final-report/src/5/img/Figure_2.png}
        \caption{Utilisation des routes suivant l'heure}
        \label{fig:road_usage_hour}
    \end{figure}

\end{frame}

\begin{frame}
    \frametitle{Résultat}
    \framesubtitle{Interprétation}

    \begin{block}{Interprétation}
        Avec de tels graphiques, les décideurs peuvent connaître les infrastructures qui doivent être étendues, ou élargies pour améliorer la mobilité des personnes de ce scénario.
    \end{block}
\end{frame}

\begin{frame}
    \frametitle{Conclusion}
        Nous avons construit une base d'outil de simulation permettant de:
    \begin{itemize}
        \item \emoji{check-mark} Mesurer l'évolution des infrastructures
        \item \emoji{check-mark} Comprendre des habitudes de transport
        \item \emoji{check-mark} Prendre en compte les activités économiques autour de la mobilité
        \item \emoji{check-mark} Prendre en compte de la singularité de la mobilité dans Cotonou
    \end{itemize}


    \begin{alertblock}{Limites}
        \begin{itemize}
            \item Non prise en compte de la qualité de routes
            \item Le simulateur est gourmand en ressouces (CPU, RAM)
        \end{itemize}
    \end{alertblock}

    \begin{block}{Perspectives}
        \begin{itemize}
            \item Implémenter les scénarios qui sont dans les projets d'urbanisation du Ministère des infrastructures et des transports (Contournement nord de Cotonou)
            \item Implémenter des scénarios de mobilité plus complexe
        \end{itemize}
    \end{block}
\end{frame}

%   On observe une efferverce
%    \begin{itemize}
%        \item Elasticsearch est un moteur de recherche basé sur la bibliothèque \href{https://en.wikipedia.org/wiki/Lucene}{Lucene}\autocite{wiki_elastic}.
%        \item Il fournit un moteur de recherche "full text" distribué, avec une interface web HTTP (api rest) et des documents JSON non structurées\autocite{wiki_elastic}.
%    \end{itemize}
%
%    \begin{block}{Régulation de traffic}
%    \end{block}

%    \begin{block}{Régulation de traffic}
%        Des systèmes de régulation existents pour assurer une bonne qualité de la mobilité.
%    \end{block}
